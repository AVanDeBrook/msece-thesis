\documentclass[10pt]{article}

\usepackage{cite}
\usepackage{hyperref}
\usepackage{listings}
\usepackage{lipsum}
\usepackage{multicol}
\usepackage{ragged2e}
\usepackage{newtxtext}
\usepackage{setspace}
\usepackage{sectsty}
\usepackage[numbib]{tocbibind}
\usepackage[lmargin=1.5in,rmargin=1.5in,tmargin=1in]{geometry}

% Conforming somewhat to the NeurIPS format
\settowidth{\parindent}{}
\setlength{\parskip}{5.5pt}
\sectionfont{\large}
\subsectionfont{\normalsize}

\begin{document}
    % Manually specify/format title, because making small formatting changes to the title is too damn frustrating
    \begin{center}
        \LARGE
        \textbf{Language Modeling of Air-Traffic Communications}\\
        \Large
        Master of Science in Electrical \& Computer Engineering\\
        Master's Thesis Proposal\\

        \vspace*{0.25in}

        \normalsize
        \textbf{Aaron Van De Brook---2456908}\\
        \textit{Department of Electrical Engineering \& Computer Science}\\
        \textit{Embry-Riddle Aeronautical University}\\
        vandebra@my.erau.edu
    \end{center}

    % An abstract. In 100 to 350 words, briefly summarize the problem to solve, the objectives of the project,
    % and the methodology to use.
    \begin{center}
        \section*{\textit{Abstract}}
        \justifying
        \textit{
            Transformer neural networks have proven to be effective in several problems domains and have seen extensive use
            in Natural Language Processing (NLP), particularly language modeling, through models such as BERT, BART, and GPT. Due
            to the success of transformers in the general NLP domain, I propose the idea of developing, training, and evaluating
            a transformer for aviation english which, despite having regulated phraseology through organizations such as ICAO,
            can take many forms including that of conversational english. The variance in the structure of the language used
            in aviation communications coupled with lack of use of transformer-based language models in the current literature
            leaves a significant gap that can be addressed by the proposed work and expanded upon in future work.
        }
    \end{center}

    % An introduction (to the problems to be addressed). In a few paragraphs, clearly define what the problem is being
    % addressed/solved in the thesis. Background information is needed. A literature review is needed to provide an
    % account of the state-of-the-art solutions of this problem by other researchers or industry experts.
    % Objectives of the thesis should be provided here as well.
    \section{Introduction}
        The transformer neural network architecture was originally designed and created for Neural Machine
        Translation (NMT) tasks \cite{vaswani_attention_2017} and has since seen massive success in various
        other Natural Language Processing (NLP) tasks such as language modeling (both casual and masked language
        modeling tasks) \cite{devlin_bert_2019}, prompt completion \cite{radford_improving_2018}, and
        sequence-to-sequence (S2S) classification \cite{lewis_bart_2019}.
        Recently, Automatic Speech Recognition (ASR) models have started to incorporate language models
        \cite{badrinath_automatic_2022} (sometimes also referred to as linguistic models) in addition to
        acoustic models \cite{li_jasper_2019} as the state-of-the-art models begin to shift towards generative
        end-to-end approaches \cite{hannun_deep_2014}.
        At the time of this writing, the current state-of-the-art ASR models are a mix between transformer
        and Convolutional neural network (CNN) architectures
        \footnote{According to \url{https://paperswithcode.com/sota/speech-recognition-on-librispeech-test-clean}}
        \cite{baevski_wav2vec_2020}.


        As ASR models begin to breakout into the aviation domain, one of the most limiting factors in their
        development and deployment has been their relatively high word error rate(s) (WER)
        \cite{smidl_air_2019,zuluaga-gomez_automatic_2020,badrinath_automatic_2022} as compared to typical state-of-the-art models.
        This is due, at least in part, to the severe lack of transcribed data as compared to the more general ASR domain.
        For instance the best estimates of combined labeled datasets for Air Traffic Control (ATC)
        communications reach approximately 180 hours of speech data \cite{zuluaga-gomez_automatic_2020} whereas
        state-of-the-art ASR datasets typically reach approximately 1000 hours of speech data \cite{panayotov_librispeech_2015}.
        This has lead to the use of semi-supervised learning methods wherein unlabeled air
        traffic communications are transcribed by a pretrained ASR model with a relatively low WER.
        The transcriptions are subsequently scored by a language model to obtain some insight into the
        transcription quality \cite{badrinath_automatic_2022,zuluaga-gomez_contextual_2021}.
        Depending on the distribution of the transcription scores, a certain percentage of the transcribed data or candidate
        samples under a specific level of uncertainty are added to the larger data set and used to retrain the model. This
        process can repeat until there is no more unlabeled data to use or until a desired WER has been reached.


        Despite the effectiveness of transformer-based architectures for language modeling NLP tasks
        \cite{devlin_bert_2019,lewis_bart_2019,liu_roberta_2019}, there is little to no implementation
        of them for modeling ATC communications for semi-supervised learning or otherwise in the
        existing literature.
        Additionally, my preliminary work into the transfer learning of BERT models in the ATC
        domain yield surprisingly high perplexity scores before overfitting to the training data.
        This suggests that a more sophisticated approach is needed to make use of transformer-based
        architectures such as BERT or RoBERTa for modeling air traffic communications.

    % TODO: formal literature review
    \section{Background \& Literature}
        % transformer history
        The transformer neural network architecture was proposed in 2017 for Neural Machine Translation tasks and immediately achieved
        state-of-the-art (28.4 BLEU on WMT 2014 English-to-German; 41.8 BLEU on WMT English-to-French datasets)
        \cite{vaswani_attention_2017}. Transformer architectures have been found to be extremely effective at learning representations
        and understandings of languages to predict token probabilities as opposed to transforming one language into another
        \cite{devlin_bert_2019,liu_roberta_2019}. Transformers have also been found to be very effective at
        other NLP-related tasks such as prompt completion and sentiment analysis among others (i.e.~auto-regressive and sequence
        classification tasks, respectively) \cite{lewis_bart_2019,radford_improving_2018}.


        % ASR context (suggest some need for semi-supervised approaches and therefore LMs)
        End-to-end generative models for automatic speech recognition models have made significant progress in recent years with
        current state-of-the-art models achieving WERs as low as 2\% on LibriSpeech test sets
        \cite{han_contextnet_2020,kriman_quartznet_2020,baevski_wav2vec_2020,li_jasper_2019}. This has led to the development of ASR
        models for the aviation domain, specifically, in air traffic control communications
        \cite{badrinath_automatic_2022,smidl_air_2019,zuluaga-gomez_automatic_2020,srinivasamurthy_semi-supervised_2017}.
        However, due to the lack of transcribed data in the aviation domain
        \cite{zuluaga-gomez_automatic_2020,srinivasamurthy_semi-supervised_2017,badrinath_automatic_2022,smidl_air_2019}
        ASR models maintain relatively high WERs compared to their counterparts in the more generalized ASR domain
        \cite{zuluaga-gomez_automatic_2020,badrinath_automatic_2022}. Transfer learning has even yielded limited results
        in this domain (depending on model architecture and dataset quality)
        \cite{badrinath_automatic_2022,zuluaga-gomez_automatic_2020}.


        % Unsupervised/semi-supervised in general and LMs commonly used with them
        Unsupervised and semi-supervised methodologies have become popular recently in attempts to address limited data availability
        and develop new approaches towards modeling human speech (notably, wav2vec has achieved state-of-the-art performance with
        very little training data)
        \cite{baevski_wav2vec_2020,badrinath_automatic_2022,srinivasamurthy_semi-supervised_2017,zuluaga-gomez_contextual_2021}.
        Language models are an integral part semi-supervised learning. They are used to obtain a certainty score (or uncertainty
        score, as the case may be) for the predicted text, these are usually either word lattices or N-gram models
        \cite{badrinath_automatic_2022,srinivasamurthy_semi-supervised_2017,zuluaga-gomez_contextual_2021}.
        While these have proven to be adequate for most self-supervised learning tasks, they are hardly state-of-the-art.
        Wav2vec, the current state-of-the-art unsupervised model (and possibly top performing overall), uses a contrastive process
        between convolutional feature extraction and transformer context representations \cite{baevski_wav2vec_2020}.


        % LMs and NLP in aero. domain
        Various natural language processing methods have been applied to the aviation domain to help deal with miscommunications
        and try to mitigate safety incidents \cite{ragnarsdottir_language_2003,tanguy_natural_2016,madeira_machine_2021}.
        Some machine learning approaches have been implemented to analyze the text in aviation incident and safety reports to predict
        contributing factors and topic models \cite{tanguy_natural_2016,madeira_machine_2021}. An ASR system with NLP post-processing
        has also been proposed to analyze Air-Traffic communications and condense significant features (e.g.~weather, runway, and
        heading info) into an XML language structure \cite{ragnarsdottir_language_2003}. Transformer language models such as BERT,
        RoBERTa, and DeBERTa have been applied to notice to airmen (NOTAM) messages to perform named entity recognition (NER) tasks,
        translation tasks (between notations; e.g.~NOTAM notation to Airlang to make parsing tasks easier) and reduce the workload
        for pilots during long-haul flights \cite{arnold_knowledge_2022}. Transformer models have also been used for speaker role
        identification tasks in the aviation domain (e.g.~identifying pilot versus controller in communications), specifically,
        a pretrained BERT model was used and fine-tuned on problem specific data and compared to other models that performed
        well at speaker and role identification tasks in general \cite{guo_comparative_2022}.

    % Methodology. Describe the approaches you plan to use for solving the problem at hand. Specify the
    % tools/instruments/procedures/designs you need to use to finish your work. Also specify other resources,
    % including the budgets, you need to use to finish your work.
    %
    % TODO: CITATIONS/REFERENCES!!
    \section{Methodology}
        This section lays out the methodology and sources for items such as data, model design, model training and testing methodology
        such as performance metrics and loss functions, preliminary data analysis and statistics, potential application areas, and
        project specific development and management techniques.

        \subsection{Data Sources \& Means of Collection}
        There are a few text datasets that exist that can be used either freely or accessed easily. There are also a few ASR datasets
        in which the primary spoken language is english (or is sorted in such a way that it is easy to isolate the english transcripts
        from the others) and the transcripts can be mined for english text for training language models. The table below lists the
        relevant datasets (hereafter referred to as \textit{the corpus}):

        % TODO: identify resources for things like NOTAMs, safety reports from relevant regulatory agencies, etc.
        %       probably not all applicable, but worth looking into regardless.
        \begin{table}[h!]
            \centering
            \begin{tabular}{c|c|c|c}
                Dataset Name & Access & Data Type & Citation \\
                \hline
                Air-Traffic Control Complete & Paid (have access) & Real, transcripts & \cite{godfrey_air_1994} \\
                ATCSpeech & Free/Paid (apply for access) & Real, transcripts & \cite{yang_atcspeech_2020} \\
                HIWIRE & Paid (have access) & Simulated, scripts & \cite{segura_hiwire_2007} \\
                ATCC & Free (have access) & Real, transcripts & \cite{smidl_air_2019} \\
                ATCOSIM & Free (have access) & Simulated, transcripts & \cite{hofbauer_atcosim_2008} \\
            \end{tabular}
            \label{table:datasets}
            \caption{Dataset name, access requirements (including prior availability through past projects), and citation information.
                    \textbf{Real/Simulated} corresponds to real world communications that were transcribed in some way or communications
                    that were recorded in a simulated environment. \textbf{Transcripts/Scripts} corresponds to real communications that were
                    transcribed or scripts that were generated.}
        \end{table}

        Since the text in the corpus comes from a variety of sources, it will have to be processed and cleaned of special characters
        or sequences specific to the original dataset. Since that process is so specific to the data source and original dataset(s),
        the specific methodology used to preprocess and clean the data will be detailed later in the thesis.

        % - primarily transformer based, since almost all state-of-the-art approaches are transformers. maybe Bayesian formulation as
        %   well, depending on how "noisy" the data is.
        % - potential optimizers to use
        % - potential loss function(s)
        % - baseline models for comparison: N-gram, word lattice, (maybe) RNN, CNN, LSTM
        \subsection{Model Design(s)}
        \textbf{Development Model(s)}. At the moment the prevailing idea for the main model architecture is the transformer
        \cite{vaswani_attention_2017}, considering its widespread success in language modeling and other NLP tasks. Language structure
        and grammar in aviation communications typically follows a very formal structure, however, deviations from this formality are
        common, for this reason it may be worth considering Bayesian Deep Learning (BDL) approach to the transformer language model
        since it has demonstrated good performance in accurately classifying data with high variance. The main tradeoffs in developing
        a BDL-transformer hybrid model would most likely be development time, and training time---from past work, end-to-end BDL models
        have taken an average of three times longer than ``normal'' networks to train.


        \textbf{Baseline Models}. These are the models against which the performance and effectiveness of language modeling, and NLP
                                           %establish?
        tasks moreover, will be compared to determine the gain in effectiveness and performance between the developed models and
        previously used models. The models that have been used in previous works in similar or identical problem domains are below:

        % TODO: the vertical alignment is aggravating; fix if possible
        \begin{multicols}{2}
            \begin{itemize}
                \item N-Gram
                \item Word Lattice
                \item RNN-based Language Models
                \item CNN-based Language Models
                \item LSTM-based Language Models
                \item Pretrained models, e.g.~BERT, RoBERTa,
            \end{itemize}
        \end{multicols}


        \subsection{Model Performance Metrics \& Evaluation}
        \textbf{Performance Metrics}.
        The common accuracy/error rate quantification metric used in language modeling is perplexity, which is typically formulated as
        the normalized inverse of the sequence probability, along with the typical classification problem metrics (i.e.~accuracy,
        precision, recall, F-1 score).


        \textbf{Loss Function(s)}.
        The most common loss function used in language modeling is Cross Entropy loss. Connectionist Temporal Classification (CTC) loss
        can also be used if the input sequence can be formulated as a time series, but this is most commonly used in ASR, since
        transcript labels are accompanied by audio time series.


        During the training and validation phases, the performance metrics will be logged each step (or epoch, whichever is more
        applicable) and plotted, most likely in TensorBoard, to keep track of performance during training. During the testing phase,
        the accuracy and average model perplexity will be reported and, if applicable, a confusion matrix will be generated and plotted
        to help illustrate the model performance.


        % uncertainty scoring, different decoding heads for things like masked language modeling, sequence classification,
        % token classification (named entity recognition), prompt completion (thinking this could be used to complete corrupted
        % text/transmissions)
        \subsection{Application Areas}
        As evidenced by the \textit{transformers} API from HuggingFace\footnote{\url{https://huggingface.co/docs/transformers}} (and the models hosted
        on their website), transformer models can be trained to a generic checkpoint through objectives such as masked language modeling to learn
        token probabilities, and fine-tuned on downstream tasks with task specific decoding layers (also commonly referred to as heads
        e.g.~sequence classification head, language modeling head, etc.). Due to this fact, the range of application areas is fairly
        extensive, ranging from casual and masked language modeling to named entity recognition (NER; also referred to as token
        classification), prompt completion, and sequence classification tasks such as sentiment analysis and infinitely more tasks and
        problems within those categories.


        There are several application areas that can be addressed by the work this thesis produces, but not all will be extensively
        tested. One of the main areas that can be addressed easily and relatively quickly is \textbf{uncertainty scoring} i.e.~using a
        LM to score transcripts generated by ASR models and determine a level of uncertainty for the validity of the transcript.
        A sequence classification task can be formed to \textbf{identify communication types} or perform \textbf{role identification}
        \footnote{There is a fair amount of research here as well, so a comparison of results wouldn't be difficult to perform.}
        (e.g.~pilot, air-controller, ground-controller, etc.; this could also be broken down into a NER task, depending on the data).

        % TODO: expand or remove (probably keep, just elaborate a bit on specifics, identify alternatives)
        \subsection{Model Development}
        Fortunately, there is a plethora of libraries available for developing machine learning, deep learning, and NLP models. The
        main packages that have been chosen (listed in Appendix A) have front-end language bindings for Python and most have
        back-end APIs written in C/C++ and Rust for streamlined computations.

    % Deliverables and timeline. Provide a detailed schedule and deliverables, shown in a table like the one below.
    \section{Deliverables \& Timeline}
        \begin{table}[h!]
            \centering
            \begin{tabular}{l|l}
                \textbf{Deliverable} & \textbf{Date} \\
                \hline
                Submit thesis proposal & October, 2022 \\
                Accumulate data sets\footnotemark[4] & December, 2022 \\
                Finish performing data analysis\footnotemark[5] & January, 2023 \\
                Finalize model designs and traning/testing methodologies for existing architectures & February, 2023 \\
                Finish developing models/scripts for training, testing models & April, 2023 \\
                Full scale training and testing of models & May, 2023 \\
                Analysis of results\footnotemark[6] & August, 2023 \\
                Full scale testing and analysis in designated application areas \& analysis of results & September, 2023 \\
            \end{tabular}
            \label{table:timeline}
            \caption{Deliverables and approximate delivery dates.}
        \end{table}
        % apparently footnotes don't work inside of float environments, so...
        \footnotetext[4]{Most likely hosted on OneDrive or SharePoint (associated with ERAU student account).}
        \footnotetext[5]{This includes relevant statistics about the data, as well as mining and converting all text into a common form.}
        \footnotetext[6]{This will include tasks such as performance analysis/comparison, generating relevant plots, and preliminary testing for
                         designated application areas.}

    \newpage
    % References. List the references in IEEE format.
    \bibliographystyle{IEEEtran}
    \bibliography{proposal_refs}
\end{document}
